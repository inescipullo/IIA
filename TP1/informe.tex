\documentclass[11pt]{article}
\usepackage[utf8]{inputenc}
\usepackage{amsfonts}
\usepackage{natbib}
\usepackage{graphicx}
\usepackage{amsmath}
\usepackage{amssymb}
\usepackage{mathrsfs} % Cursive font
\usepackage{graphicx}
\usepackage{ragged2e}
\usepackage{fancyhdr}
\usepackage{nameref}
\usepackage{wrapfig}
\usepackage{hyperref}


% \usepackage[left=2cm, right=2cm, top=2cm,bottom=2cm]{geometry}


\usepackage{mathtools}
\usepackage{xparse} \DeclarePairedDelimiterX{\Iintv}[1]{\llbracket}{\rrbracket}{\iintvargs{#1}}
\NewDocumentCommand{\iintvargs}{>{\SplitArgument{1}{,}}m}
{\iintvargsaux#1}
\NewDocumentCommand{\iintvargsaux}{mm} {#1\mkern1.5mu,\mkern1.5mu#2}

\makeatletter
\newcommand*{\currentname}{\@currentlabelname}
\makeatother

% \usepackage[a4paper,hmargin=1in, vmargin=1.4in,footskip=0.25in]{geometry}

\graphicspath{ {./images/} }


%\addtolength{\hoffset}{-1cm}
%\addtolength{\hoffset}{-2.5cm}
%\addtolength{\voffset}{-2.5cm}
\addtolength{\textwidth}{0.2cm}
%\addtolength{\textheight}{2cm}
\setlength{\parskip}{8pt}
\setlength{\parindent}{0.5cm}
\linespread{1.5}

\pagestyle{fancy}
\fancyhf{}
\rhead{TP1 - Bisiach, Cipullo, Sullivan}
\lhead{Introducción a la Inteligencia Artificial}
\rfoot{\vspace{1cm} \thepage}

\renewcommand*\contentsname{\LARGE Índice}

\setlength{\skip\footins}{0.5cm}

\begin{document}

\begin{titlepage}
    \hspace{-2.5cm}\includegraphics[scale= 0.48]{header.png}
    \begin{center}
        \vfill
            \noindent\textbf{\Huge Introducción a la Inteligencia Artificial}\par
            \vspace{.5cm}
            \noindent\textbf{\Huge Trabajo Práctico 1: Algunas aplicaciones interesantes de la IA}\par
            \vspace{.5cm}
        \vfill
        \noindent \textbf{\huge Alumnos:}\par
        \vspace{.5cm}
        \noindent \textbf{\Large Bisiach, Ezequiel}\par
        \noindent \textbf{\Large Cipullo, Inés}\par
        \noindent \textbf{\Large Sullivan, Katherine}\par
 
        \vfill
        % \large Universidad Nacional de Rosario \par
        \noindent\large 2022
    \end{center}
\end{titlepage}
\ \par


\section{Introducción}
Tal vez es por su tratamiento en la ficción, tal vez es 
por la manera de pensar del público en general pero no
cabe ninguna duda de que una de las preguntas que 
automáticamente se asocia con la inteligencia artificial
es: ``¿pueden las máquinas reemplazar a los humanos?". 

Si 
bien esta pregunta abre un debate más amplio que no es el foco
de lo 
que será discutido en el presente informe, sí la revisitaremos y es digna 
su
mención puesto que a continuación se expondrá sobre
Woebot, una inteligencia artificial que,
en un principio, se podría decir
que busca reemplazar
a personas, pero a un tipo específico de personas: 
terapeutas conginitivo conductuales. 

% El presente informe se enfocará en la aplicación de la inteligencia artificial en la terapia cognitava conductual (CBT por sus siglas en inglés \textit{Cognitive Behavioral Therapy}), en particular en la herramienta Woebot. Woebot es un chatbot que intenta simular un ``terapeuta artificial''. El uso de la inteligencia artificial en la psicología abre distintas puertas de debate, las cuales se procederán a desarrollar.

%%% \subsection: pq CBT y no psicoanalisis u otras terapias, en esta subsection hablaría tambien del punto 2)

\section{Sobre el propósito de Woebot}
Woebot es un chatbot que busca garantizar el acceso a la terapia 
cognitivo conductual (CBT por sus siglas en inglés) a 
un rango más extenso de personas (entre ellas, personas 
sin los suficientes recursos económicos\footnotemark[1]\footnote[1]{Hoy la aplicación de Woebot es gratuita 
al recibir finanaciamiento a través de un fondo de 
capital emprendedor. Pero según establecido por su 
fundadora se apunta a que algún día la aplicación 
pueda ser recetada y por lo tanto pagada como un 
tratamiento, no sin perder de vista su objetivo de 
hacerla accesible a todos aquellos que necesiten la 
herramienta.} o personas con 
patologías afines a la ansiedad social) y en un rango 
de horario irrestricto (lo que resulta especialmente 
beneficioso para el control de episodios patólogicos 
que se pueden dar en cualquier momento). 

% Por el otro, cuenta con la ventaja de ser un servicio disponible las 24 horas, lo que resulta especialmente benefisioso para aquellas personas que requieran de acompañamiento terapéutico sin restricciones horarias. 

\section{Sobre las técnicas utilizadas por Woebot}
%Resulta de especial interés para la materia ver qué técnicas de IA usa esta aplicación. 
Haciendo uso del procesamiento del lenguaje natural (NLP de sus siglas en inglés 
\textit{Natural Language Processing}), la aplicación 
logra simular el manejo del lenguaje como una persona. 
Es decir que puede analizar los mensajes recibidos 
capturando 
palablas clave 
que determinarán su semántica
y posteriormente aplicar los
algoritmos correspondientes para generar una respuesta.

% Para realizar el procesamiento del lenguaje natural se hacen uso de redes neuronales

\section{Sobre los estudios realizados}
El grupo que desarrolla Woebot es Woebot Health, un grupo interdisciplinario que conjuga especialistas en psicoterapia y especialistas en inteligencia artificial fundado y dirigido por Alison Darcy, doctora de la Universidad de Standford.

Diversos estudios fueron publicados (muchos de ellos contando con colaboración del Departamento de Psiquiatría de la Escuela de Medicina de la Universidad de Standford) demostrando la eficiencia de Woebot para ayudar en cuadros generales de depresión y ansiedad, en casos de abuso de sustancias y en cuadros de depresión post-parto.
Referencias a los mismos y resúmenes sobre sus resultados 
junto con otros artículos que podrían resultar de interés
se pueden encontrar en el \href{https://woebothealth.com/img/2021/05/Woebot-Health-Research-Bibliography_May-2021-1-1.pdf}{Woebot Health Research Book}.

\section{Sobre las implicancias de Woebot}

La psicología requiere un tratamiento más que sensible
con el paciente en general pero también con sus datos.
El secreto profesional es algo que se da por sentado en
una sesión terapéutica con un profesional pero, ¿dónde cae
el secreto en un programa que necesita guardar explícitamente
los datos del paciente y todo lo que este le comunica
para poder aprender de ellos? No necesariamente se desconfía
de la divulgación de los mismo por parte de la aplicación
pero que información tan sensible esté guardada y propensa a
un hackeo es algo que no se puede dejar de lado al momento
de analizar este tipo de aplicaciones.

Pero esa no es la única implicancia que surge relacionada
con el tratamiento de los datos y Woebot. Históricamente 
existieron sesgos en los diagnósticos psicológicos por 
trabajar con datos que no consideraban la diversidad de
grupos, uno de los casos más clásicos siendo la 
investigación sobre los trastornos del espectro autista
(TEA)
cuyo foco puesto solo en hombres a día de hoy sigue 
generando que el determinar el diagnóstico de una 
mujer con TEA sea muchísimo más difícil que uno
correspondiente a un varón. Hace pocos años que 
se le está prestando atención a estos hechos y 
que aparezcan máquinas
en la ecuación solo puede generar más problemas a la 
hora de tener en cuenta estos sesgos. 

Sin embargo, no todas las implicancias fuera de su propósito
hacen de Woebot algo indeseable, si no, ¿por qué estaríamos 
hablando de él? Puede ser que sea un deseo muy futurístico,
pero se cree que tal vez algún día tanto desarrollo de 
inteligencias que busquen ser infalibles ayuden a que los
humanos seamos mejores, y en este caso en particular que 
tanto modelizado de terapia nos pueda ayudar a crear 
mejores técnicas y mejores profesionales.


%%% Esta sección podria ser la parte de apreciaciones personales, pero habria que agregarle alguna cosa positiva que contrareste toda la critica xd, sino se puede acomodar el algun otro lado
\section{Conclusiones y un pequeño análisis sobre las cuestiones a resolver si queremos un futuro con IAs}
Uniendo un poco lo desarrollado en la introducción y
lo planteado en esta última sección podemos ver que la 
idea por la que se está trabajando en el campo de la 
inteligencia artificial puede parecer que es 
reemplazar a los humanos pero la realidad es que se busca
que las intelgencias artificiales reemplacen a humanos en 
ciertas tareas y que nosotros podamos progresar en base a 
ellas, focalizándonos en áreas en donde se seguirá 
necesitando de inteligencia humana y, principalmente, 
aprendiendo de lo que pudimos enseñar.

Ahora bien, no hay que perder de vista que se necesita 
de un esfuerzo aunado para establecer
tanto marcos morales como legales que permitan el desarrollo 
metódico y ordenado
de las inteligencias que queremos. El desarrollo de 
comisiones internacionales que regulen las 
prácticas de inteligencia artificial son un buen presagio 
pero verdaderamente necesitamos un diseño estructural 
de la sociedad que desarrollará y usará inteligencias 
artificiales en su día a día. El más tarde no es
para lamentarse si tenemos el ahora.


%Volviendo a lo planteado en la introducción, 
%si bien resulta claro que un programa no logra emular una persona, 
%al menos actualmente, sí se busca que pueda semejarse a ellas en ciertos aspectos, más cuando interactúan directamente con humanos en un ámbito sensible como es la psicología.
% Dicho esto, resulta importante la necesidad de que estos terapeutas artificiales puedan realizar tareas como proveer soluciones en momento críticos, acompañar emocionalmente y hasta saber cual es el límite de sus "capacidades" y poder derivar a las personas a otras fuentes de ayuda.

%Por otro lado, surgen algunas cuestiones legales que se tienen que regular. Se debería establecer una cadena de responsabilidades sobre personas para casos de mala praxis y demás responsabilidades civiles en el ejercicio de la psicología, que no necesariamente van a recaer sobre el grupo de desarrolladores que armaron el software de la aplicación. Además, como el programa va a almacenar toda la información que intercambie con las distintas personas, en parte como registro pero también para continuar con su "formación", habría que determinar como se maneja el secreto profesional y quienes tienen acceso a ello. Muchos argumentarán que no es ético que se reocpilen todos esos datos personales.

%%% faltaria hablar de los sesgos, punto 6)
\newpage

\section{Bibliogarfía}

\begin{enumerate}

\item Darcy A., Daniels J., Salinger D., Wicks P., Robinson A.:
\textit{Evidence of Human-Level Bonds Established With a Digital Conversational Agent: Cross-sectional, Retrospective Observational Study},
URL: \url{https://formative.jmir.org/2021/5/e27868},
DOI: 10.2196/27868

\item Ho A., Hancock J., Miner A.:
\textit{Psychological, Relational, and Emotional Effects of
Self-Disclosure After Conversations With a Chatbot},
URL: \url{https://www.ncbi.nlm.nih.gov/pmc/articles/PMC6074615/}

DOI: 10.1093/joc/jqy026

\item Martínez-Miranda J.: 
\textit{Embodied Conversational Agents for the Detection and 
Prevention of Suicidal Behaviour: 
Current Applications and Open Challenges},
URL: \url{https://europepmc.org/article/med/28755270}
DOI: 10.1007/s10916-017-0784-6

\item Miner A., Chow A., Adler S., Zaitsev I., Tero P., Darcy A., and Paepcke A.:
\textit{Conversational Agents and Mental Health: Theory-Informed Assessment of Language and
Affect. In Proceedings of the Fourth International Conference on Human Agent Interaction (HAI
'16)},
URL: \url{https://doi.org/10.1145/2974804.2974820}
DOI: 10.1145/2974804

\item Prochaska J., Vogel E., Chieng A., Kendra M., Baiocchi M., Pajarito S., Robinson A.:
A Therapeutic Relational Agent for Reducing Problematic Substance Use (Woebot): Development and Usability Study, 
URL: \url{https://www.jmir.org/2021/3/e24850},
DOI: 10.2196/24850
\end{enumerate}
\end{document}